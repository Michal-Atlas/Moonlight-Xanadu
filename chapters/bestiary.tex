\chapter{The DawnStorm Bestiary}

\hypertarget{recommended-playable-humanoids}{%
\subsection{Recommended Playable
Humanoids}\label{recommended-playable-humanoids}}

\hypertarget{human}{%
\subsubsection{Human}\label{human}}

Nothing special, the perfect blank slate.

\hypertarget{goblin}{%
\subsubsection{Goblin}\label{goblin}}

Size: \texttt{Small}

\hypertarget{keinfolk}{%
\subsubsection{Keinfolk}\label{keinfolk}}

\textbf{Kein Roar} - Once a day, you are able to Roar so loud that any
creatures within a 30 meter radius, must succeed at an \texttt{RD\ 9}
Check, or become \texttt{Frightened} of you for 6 Seconds.

Incredibly proud, stubborn and loyal creatures, described mostly as
humanoid lions. Pride upon pride, their mane is what defines their
class, long and wavy manes always giving an advantage when arriving in
any Keinfolk society.

\hypertarget{talach}{%
\subsubsection{Talach}\label{talach}}

\textbf{Talach talent} - You are a \texttt{Master} at one form of art
and \texttt{Skilled} in another.

Tall, with pointed ears, humble and noble, truly a beloved race by most
anyone who meets them. Their curiosity and fantasy had made them into
popular artists and poets, outside their cities often kept in castles
and manors by nobility due to this affinity. Their intelligence hasn't
been perceived to be high nor their strength, manual dexterity is the
one thing they have and which no other race may rival them in.

\hypertarget{talkin}{%
\subsubsection{Talkin}\label{talkin}}

\textbf{Talkin Ear} - Your ears are very sensitive to whispers and
murmurs. When listening to someone in a crowded room, you hear as if the
room was silent.

They are very similar to the Talach, however their ears are
significantly flatter, lower, sometimes almost horizontal. Quite sly,
sneaky and adept at close range combat, they make excellent spies and
thieves and are often employed as such.

\hypertarget{tylluan}{%
\subsubsection{Tylluan}\label{tylluan}}

\textbf{Tylluan Sight} - You have \texttt{NightVision}

The Tylluan are to owls what the keinfolk are to lions.

\hypertarget{beasts}{%
\subsection{Beasts}\label{beasts}}

\hypertarget{warthorg}{%
\subsubsection{Warthorg}\label{warthorg}}

\hypertarget{celestials}{%
\subsection{Celestials}\label{celestials}}

\hypertarget{angel}{%
\subsubsection{Angel}\label{angel}}

\hypertarget{demons}{%
\subsection{Demons}\label{demons}}

\hypertarget{bone-demons}{%
\subsubsection{Bone Demons}\label{bone-demons}}

The bone demons are the knights of the nine hells, they are sent to
retrieve souls that are expected to pose ressistance. They stand about 2
metres tall, carry a great-shield and a slightly curved broadsword or
lance, as well as two daggers lodged into a cavity in their hips. Their
outer appearance is that of a knight, whose plate armour has a surface
similair to a bones. It's head has a fleshy mouth, and then a split into
two flat protrusions to each side. If the bone demon's plates should
come apart you may see that the inside is actually mostly empty, and
held together by occassional oozing strands, the only physical part
being the mouth.

\hypertarget{dreadlords}{%
\subsubsection{Dreadlords}\label{dreadlords}}

\hypertarget{nightcrawler}{%
\subsubsection{Nightcrawler}\label{nightcrawler}}

These dog-like creatures are mostly used by other demons for spying,
hunting and disposing of targets. Their visage is quite simple, the
front half is that of a wolf up until their back legs and hips, which
are just black smoke and non-visible. Their eyes reflect light in
anything but the darkest of places, the light changes color based on the
demon it is currently serving. If a mortal ever manages to calm and tame
a nightcrawler, the smoke coming from its back changes to bright blue
and it becomes apparent that their non-feral nature is quite playful.

\textbf{Blinking} They can teleport between two points connected by near
pitch darkness while neither is being observed.

\hypertarget{oni}{%
\subsubsection{Oni}\label{oni}}

\hypertarget{fey}{%
\subsection{Fey}\label{fey}}

\hypertarget{changelings}{%
\subsubsection{Changelings}\label{changelings}}

Changelings are the infiltrators of the Fey, they can change their body
to the visage of any humanoid they see at least indirectly. They are
born by imbuing an unborn child with Fey magic. The family usually
doesn't know about this, until the Fey visit the child at 18 years old,
informing it of its destiny. This arduous process is mostly so that the
Changeling blends in as well as possible with other creatures. When a
changeling dies, it reverts to its original form.

\hypertarget{dryad}{%
\subsubsection{Dryad}\label{dryad}}

\hypertarget{fey-lord}{%
\subsubsection{Fey Lord}\label{fey-lord}}

\hypertarget{fey-queen}{%
\subsubsection{Fey Queen}\label{fey-queen}}

\hypertarget{krypsaker}{%
\subsubsection{Krypsaker}\label{krypsaker}}

A fey being of intense resilience, it resembles a humanoid, but it seems
as to be disfigured into a goblin-like form, with sharp teeth and
pointed uneven ears and nails. At any point it can turn parts of itself
into insects or birds and Even a single one of these may with enough
nourishment create a whole Krypsaker once more.

\hypertarget{pan-pixie}{%
\subsubsection{Pan / Pixie}\label{pan-pixie}}

The most common of the Fey, often incorrectly called forest nymphs, Pans
and their female counterparts called Pixies inhabit kingdoms in many
forests of the world. Their philosophy of life is inherently pacifistic,
created by the titan gaya to protect the forests, the damage of which is
the only way to aggravate them.

They carry human chests and heads, usually with antlers and animal eyes
and ears. Their legs bend backwards above the knee bending backwards
before heading to the ground once more and ending in hooves. Most of
their bodies are covered and seem to be made of leaves, branches and
vines, same as their tools and homes. Being gaya's chosen all wildlife
listens and bends to the needs of the pan.

\hypertarget{shadow-pan}{%
\paragraph{Shadow Pan}\label{shadow-pan}}

\hypertarget{hag}{%
\paragraph{Hag}\label{hag}}

\hypertarget{ghosts}{%
\subsection{Ghosts}\label{ghosts}}

\hypertarget{animated-armour}{%
\subsubsection{Animated Armour}\label{animated-armour}}

\hypertarget{genius-loci}{%
\subsubsection{Genius Loci}\label{genius-loci}}

These peculiar demi-ghosts usually manifest in places of great
historical or natural significance that have nonetheless been left to
their own devices. These creatures are usually very shy and rarely ever
speak unless somehow directly invoked. They have been shown to have
knowledge on all matters occurring in their place of existence. These
plains of land where they exist may range from a single ruin to most of
a forest. They communicate telepathically with any sentient creatures
within their domain, however to do this they must concentrate and so
they loose sight of their surroundings for the time being. A very
powerful Genius Loci was once recorded to speak audibly directly through
a human, both of whom shortly lost consciousness afterwards, however
this is only hearsay.

\hypertarget{rage}{%
\subsubsection{Rage}\label{rage}}

\hypertarget{spectra}{%
\subsubsection{Spectra}\label{spectra}}

\hypertarget{wendigo}{%
\subsubsection{Wendigo}\label{wendigo}}

\hypertarget{will-o-wisp}{%
\subsubsection{Will-o'-Wisp}\label{will-o-wisp}}

\hypertarget{demi-ghosts}{%
\subsubsection{Demi-Ghosts}\label{demi-ghosts}}

\hypertarget{lost-soul}{%
\paragraph{Lost Soul}\label{lost-soul}}

Lost Souls is a general term for souls that have been prevented from
leaving the world, usually through trapping them inside a container,
very commonly glass spheres or stones. When touching the container one
may telepathically communicate with the soul trapped within. It is said
that the feeling of being a lost soul is like being restrained and
gagged, however you see everything that goes on around you. If left
alone for long enough the soul usually goes crazy within a few weeks,
thus they always wish to be released and will trade next to any help
they can provide for you letting them go. Cracking or opening a
container of the soul, releases it.

\textbf{Magic} Lost souls are often used by spellcasters, because the
mana of the original creature still resides within their soul. The soul
however, cannot regenerate mana without the body and so is one use only.
\emph{Note} Please break your soul stones after use.

\hypertarget{mirra}{%
\paragraph{Mirra}\label{mirra}}

Mirras are perfect copies of a creature's conciousness from a certain
point. They are created via a spell etched into the wooden part of a
mirror, binding a present creature to the object, which then shows in
its reflection the bound creature's reflection. The reflection moves
around the room and can talk and recall all the memories of the original
creature, its maneurism and personality are also the same. The Mirra
cannot create new memories or create new impressions on people, it is
always just a perfect copy, however it may utilize very short term
memory or connect concepts to changes in the room which it may ask you
to make. The Mirra recalls existing memories perfectly and
instantaneously.

\hypertarget{magical-beasts}{%
\subsection{Magical Beasts}\label{magical-beasts}}

\hypertarget{chimera}{%
\subsubsection{Chimera}\label{chimera}}

\hypertarget{griffin}{%
\subsubsection{Griffin}\label{griffin}}

\hypertarget{lycanthropes}{%
\subsubsection{Lycanthropes}\label{lycanthropes}}

\hypertarget{night-mare}{%
\subsubsection{Night-mare}\label{night-mare}}

\hypertarget{unicorn}{%
\subsubsection{Unicorn}\label{unicorn}}

\hypertarget{titans}{%
\subsection{Titans}\label{titans}}

\hypertarget{aegir}{%
\subsubsection{Aegir}\label{aegir}}

\hypertarget{atlas}{%
\subsubsection{Atlas}\label{atlas}}

\hypertarget{cronus}{%
\subsubsection{Cronus}\label{cronus}}

\hypertarget{gaya}{%
\subsubsection{Gaya}\label{gaya}}

\hypertarget{prometheus}{%
\subsubsection{Prometheus}\label{prometheus}}

\hypertarget{sekhmet}{%
\subsubsection{Sekhmet}\label{sekhmet}}

\hypertarget{leviathans}{%
\subsubsection{Leviathans}\label{leviathans}}

\hypertarget{plants}{%
\subsection{Plants}\label{plants}}

\hypertarget{reptiles}{%
\subsection{Reptiles}\label{reptiles}}

\hypertarget{basilisk}{%
\subsubsection{Basilisk}\label{basilisk}}

A Basilisk is a solitary eight legged reptile-like creature. At first it
walks on all eight, later in life however, when its body is strong
enough to support this development, it's head and front limbs move
upwards and elongate, eventually creating an upright upper body.

\textbf{Growth Cycle} The most magnificent property of the Basilisk is
its growth cycle. A Basilisk gives birth to a live baby no larger than a
few inches. The Basilisk then grows inexplicably rapidly as the Basilisk
consumes massive amount of sustenance. At the size of about triple that
of a human the Basilisk develops an upright upper body and loses much of
the potency of its breath. When it reaches such a size that its legs are
struggling to support the body, it migrates to a large body of water.
Even if it hunts enough food, which is rarely the case, its size forces
it to continuously move to ever deeper water, eventually crushing it.

\textbf{Breath} The legends often speak of the Basilisk's noxious
breath. It has been observed to have the ability to wither small plants
and to accelerate corrosion of metal. Any effects beyond this are empty
claims.

\textbf{Gaze} \ldots{}

\hypertarget{dragon}{%
\subsubsection{Dragon}\label{dragon}}

Every dragons has the ability to transform into a humanoid.

\hypertarget{wyvern}{%
\subsubsection{Wyvern}\label{wyvern}}

Wyverns are very feral animals, true beasts fully devoted to that state
of behaviour. None have ever been recorded to reason or speak unlike
dragons, this is very important to know for any adventurer that may by
any chance encounter this menace. The distinguishing feature is that it
has only four limbs, two legs and two wings, and is inherently
non-magical. At the end of each wing it has claws that it may use to
clamber up surfaces. They come in many forms and colors, though most
often seen in the black form, which is also the most adept at climbing.
Unlike dragons, most Wyverns do not possess a fiery breath, only the red
kind is famously capable of this feat.

\textbf{Eggs} Their eggs grow in size as the Wyvern inside matures,
being laid at about half their final size. For this reason the surface
is much softer than one would at first assume, and if even the softness
of the shell is not enough the egg may crack at regular intervals
exposing the inner membrane.

\textbf{Weaknesses} You may be tempted to slash the underside of the
head, however unlike dragons, Wyverns have solid armour covering the
entire head. The two weakest points are their thigh joints and the point
where their patagia connect to the body as it has been discovered to
easily tear in the first section. When the wings are folded however the
patagia is relaxed, elastic and quite difficult to pierce without an
exceptionally sharp weapon.

\hypertarget{sgrechian}{%
\paragraph{Sgrechian}\label{sgrechian}}

This breed of Wyverns is extremely rare. Usually artificially enchanted
for combat purposes. They naturally have two main characteristics, their
speed, and their scream. Though lacking a beath, they are able to
surpass the sound barrier, by a tiny amount. Their head is often
enchanted with a spell that protects them from impact, it activates with
their scream. Their scream can be heard miles around and has deafened
many who were too close, it is described as being extremely
high-pitched, yet still powerful enough to be felt deep in the guts.

\hypertarget{undead}{%
\subsection{Undead}\label{undead}}

\hypertarget{ghoul}{%
\subsubsection{Ghoul}\label{ghoul}}

\hypertarget{mummies}{%
\subsubsection{Mummies}\label{mummies}}

\hypertarget{haugbui}{%
\paragraph{Haugbui}\label{haugbui}}

\hypertarget{vampire}{%
\subsubsection{Vampire}\label{vampire}}

\hypertarget{zombie}{%
\subsubsection{Zombie}\label{zombie}}
